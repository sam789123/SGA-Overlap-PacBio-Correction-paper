\chapter{Introduction}


In recent years, next generation sequencing (NGS) technologies, including the 454 pyrosequencing~\cite{Marcel2005} in 2004 and Illumina sequencing-by-synthesis~\cite{Bentley2006}, have been widely used to sequence and assemble genomes of many species~\cite{NGS2010, genome10k2009}. NGS technologies, which have revolutionized DNA sequencing by diminishing cost and raising throughput, can yield $10^{8}$ short reads of length up to 200 bp with low error rates (2\%). 

%-> but the repetitive sequences may up to 1,300 bp
Regardless of benefit, NGS technologies still have several drawbacks. First, NGS technologies base on the Multiple Displacement Amplification (MDA) technology, leading to amplification artifacts~\cite{pmid24852006} and extremely non-uniform coverage of the genome~\cite{pmid21685062}. Secondly, because of the complex repeated sequences in genomes~\cite{pmid18341692}, the read length yielded by NGS are insufficient to cross the repeat region. For instance, the Illumina platforms generate reads up to 150 bp~\cite{pmid23644548} and the 454 sequencing yields reads of ~700 bp~\cite{pmid23644548} but the repetitive sequences may up to 1,300 bp~\cite{pmid25132181}. Therefore, numerous genomes sequenced and assembled by pure NGS are still too fragmented for subsequent analysis, such as gene annotation. 


Recently, the third generation sequencing (TGS) technology yields much longer reads in comparison with NGS. For example, the Pacific Biosciences (PacBio) sequencing technology generates reads up to ~20 kbp (average ~3 kbp)~\cite{pmid25705213}. In addition, because the entire sequencing protocol is single molecular, it requires no amplification and substantially reduces the sequencing bias~\cite{pmid20858600}. Hence, the longer reads make it useful for scaffolding de novo assemblies. Recently, the successful assembly of $E. coli K12$, $B. trehalosi$ and $M. haemolytica$ genomes have demonstrated the power of TGS~\cite{Koren2013}.

%Therefore-> Later
In reality, these TGS reads tend to have high error rates (up to 15\%) in comparison with NGS. On the basis of high-coverage PacBio sequencing, PBcR is able to self-correct the PacBio reads and assemble the genome. However, the cost of TGS is still very expensive, and high-coverage TGS is often not affordable in large genome sequencing projects. On the other hand, a few studies have been endeavoring to correct TGS reads using NGS reads. For example, the LoRDEC~\cite{Salmela2014} and the PacBioToCa~\cite{Koren2012}. The former uses bloom filters to build a de Bruijn graph that stands for NGS reads and corrects the erroneous region from corrective sequences in TGS reads by traversing the suitable paths in the graph. The latter corrects TGS reads individually by mapping NGS reads to them and computes high identity of hybrid consensus sequences. Later, TGS reads are able to be corrected by the NGS reads which aligns on TGS reads.


In this thesis, we introduce a error correction approach that utilizes NGS reads, high-identity sequences to correct TGS reads with the internal error. In particular, we use compressed data structures named FM-index~\cite{pmid20529929} that operate over a compressed representation of the full set of NGS reads since we require substantially lower the amount of memory to perform. Next, we align high-identity NGS reads on low-accuracy TGS reads and in our expectation, TGS reads can be modified based on the information from aligned NGS reads. Eventually, the corrected TGS reads would have much lower error rates than original TGS reads. As demonstrated below for several genomes, including 4.6 Mbp genome size of $Escherichia coli K12$ and 100.2 Mbp genome size of $C elegans$, we will evaluate the performance according to the yield (the number of TGS read aligned on reference genome after correction), accuracy rates after correction, memory usage, and the running time for each species by each software which we mentioned above and our method.